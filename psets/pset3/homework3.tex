\documentclass[11pt]{article}   
\usepackage{fullpage}
\usepackage{amsfonts}
\usepackage{amssymb}
\usepackage{amsmath}
\usepackage{xcolor}

\newcommand{\F}{\mathbb{F}}
\newcommand{\np}{\mathop{\rm NP}}
%\newcommand{\binom}[2]{{#1 \choose #2}}
\newcommand{\Z}{{\mathbb Z}}
\newcommand{\vol}{\mathop{\rm Vol}}
\newcommand{\conp}{\mathop{\rm co-NP}}
\newcommand{\atisp}{\mathop{\rm ATISP}}
\renewcommand{\vec}[1]{{\mathbf #1}}
\newcommand{\cupdot}{\mathbin{\mathaccent\cdot\cup}}
\newcommand{\mmod}[1]{\ (\mathrm{mod}\ #1)}  

\setlength{\parskip}{\medskipamount}
\setlength{\parindent}{0in}
%\input{dansmacs}


\begin{document}
	
	\section*{CS 121 Homework 3: Fall
		2020}\label{cs-121-homework-zero-fall-2020}
	
	\textbf{Some policies:} (See the course policy at  {\tt http://madhu.seas.harvard.edu/courses/Fall2020/policy.html} for the
	full policies.)
	
	\begin{itemize}
		\item
		{\bf Collaboration:} You can collaborate with other students that are currently enrolled in
		this course (or, in the case of homework zero, planning to enroll in
		this course) in brainstorming and thinking through approaches to
		solutions but you should write the solutions on your own and cannot
		share them with other students. 
		\item
		{\bf Owning your solution:} Always make sure that you ``own'' your solutions to this other problem
		sets. That is, you should always first grapple with the problems on
		your own, and even if you participate in brainstorming sessions, make
		sure that you completely understand the ideas and details underlying
		the solution. This is in your interest as it ensures you have a solid
		understanding of the course material, and will help in the midterms
		and final. Getting 80\% of the problem
		set questions right on your own will be much better to both your
		understanding than getting 100\% of the questions through
		gathering hints from others without true understanding.
		\item
		{\bf Serious violations:} Sharing questions or solutions with anyone outside this course,
		including posting on outside websites, is a violation of the honor
		code policy. Collaborating with anyone except students currently
		taking this course or using material from past years from this or
		other courses is a violation of the honor code policy.
		\item
		{\bf Submission Format:} The submitted PDF should be typed and in the same format and
		pagination as ours. Please include the text of the problems and write
		\textbf{Solution X:} before your solution. Please mark in gradescope 
		the pages where
		the solution to each question appears. Points will be deducted if you
		submit in a different format.
		\item {\bf Late Day Policy:} To give students some flexibility to manage your schedule, you are allowed a net total of {\bf eight} late days through the semester, but you may not take more than {\bf two} late days on any single problem set. No exceptions to this policy. 
	\end{itemize}
	
	\textbf{By writing my name here I affirm that I am aware of all policies
		and abided by them while working on this problem set:}
	
	\textbf{Your name:} (Write name and HUID here)
	
	\textbf{Collaborators:} (List here names of anyone you discussed
	problems or ideas for solutions with)
	
	% For homeworks 1 through n
	\textbf{No. of late days used on previous psets (not including Homework Zero): }\\
	\textbf{No. of late days used after including this pset: }
	
	
	\newpage
	
	\subsection*{Questions}\label{questions}
	
Please solve the following problems. Some of these might be harder than
the others, so don't despair if they require more time to think or you
can't do them all. Just do your best. Also, you should only attempt the
bonus questions if you have the time to do so. If you don't have a proof
for a certain statement, be upfront about it. You can always explain
clearly what you are able to prove and the point at which you were
stuck. 
%Also, for a non bonus question, 
You can always simply write
\textbf{``I don't know''} and you will get 15 percent of the credit for
this problem. If you are stuck on this problem set, you can use Piazza to
send a private message to all staff.

\iffalse{ \textbf{Note on ``bonus'' questions:} There is no difference between
``bonus'' points and regular points. The indication ``bonus'' is just a
helpful hint for you: it means that the question might be harder, and
you might not want to spend inordinate amounts of time on it. Also, the
total point value of all non bonus questions will always be at least
100, so you can get a perfect grade on the problem set component of this
course without ever solving a bonus question.

\textbf{Note on reading the textbook:} If you are stuck on some of the
problems, try consulting the book to 1) understand the concepts the
question is referencing, and 2) review the way similar theorems are
proved in the book. Some helpful sections of the book for this problem
set are: Chapter 8 (for Rice's Theorem), Chapter 10 (to see solved
reductions to \(HALT\) and \(HALTONZERO\)), and Chapter 9 (for proofs
utilizing the pumping lemma and other useful proofs regarding the
computability/halting of regular expressions).

}\fi 

\newcommand{\N}{\mathbb{N}} 
\newcommand{\XOR}{\mathrm{XOR}} 
\newcommand{\Size}{\mathrm{SIZE}} 

\textbf{Problem 1:}
Recall that in section, we defined the function $LOOKUP_k: \{0,1\}^{2^k + k} \to \{0,1\}$ by 
$$LOOKUP_k(X[0], X[1], \ldots, X[2^{k}-1], i[0], \ldots, i[k-1]) = X[i[0] + 2 i[1] + \cdots + 2^{k-1}i[k-1]]$$ 
and proved that there exists a circuit with $O(2^k)$ gates that computes $LOOKUP_k$. We'll write $LOOKUP_k(X, i)$ as shorthand for $LOOKUP_k(X[0], X[1], \ldots, X[2^{k}-1], i[0], \ldots, i[k-1])$.

\textbf{Problem 1.1 (8 points)} Prove that $LOOKUP_k(X,i)$ cannot be computed by circuits $C$ of size $o(2^k)$.

\textbf{Solution 1.1:}% (write your solution here)

\textbf{Problem 1.2 (4 points)} Prove that for every $X \in \{0,1\}^{2^k}$ there exists a circuit $C(X)$\footnote{That is, the circuit may depend on $X$.} of size $o(2^k)$ that computes $LOOKUP_k(X,i)$. (You may use results stated but not proved in lecture.)

\textbf{Solution 1.2:}% (write your solution here)

\textbf{Problem 1.3 (8 points)} Prove that there exists $X \in \{0,1\}^{2^k}$ such that there does not exist a circuit $C(X)$\footnotemark[1] of size $o(2^{\sqrt k})$ that computes $LOOKUP_k(X,i)$.

\textbf{Solution 1.3:}% (write your solution here)

\newcommand{\BSize}{\mathrm{BSIZE}}

\textbf{Problem 2:}
Recall that 
$$\Size_{n,m}(s) = \{f:\{0,1\}^n \to \{0,1\}^m | \exists \mbox{ NAND circuit $C$ of at most $s$ gates s.t. $C$ computes $f$ }\}.$$ 
Let $\BSize(s) = \cup_{n \in \N}  \Size_{n,1}(s)$ represent the set of Boolean functions (i.e., those whose output is in $\{0,1\}$) that are computable by NAND circuits of size at most $s$.

Prove the following:

\textbf{Problem 2.1 (3 points):} $|\BSize(s)| =  2^{O(s \log s)}$. 

\textbf{Solution 2.1:}% (write your solution here)

\textbf{Problem 2.2 (15 points):} $|\BSize(s)| = 2^{\Omega(s)}$.

\textbf{Solution 2.2:}% (write your solution here)

\textbf{Problem 2.3 (2 points):} $\BSize(s) \subsetneq \BSize(s^2)$, for every sufficiently large $s$.  

\textbf{Solution 2.3:}% (write your solution here)

\iffalse{  % Version 2 - revised to fix problem with size definition
\textbf{Problem 2:}
Recall that 
$$\Size_{n,m}(s) = \{f:\{0,1\}^n \to \{0,1\}^m | \exists \mbox{ NAND circuit $C$ of at most $s$ gates s.t. $C$ computes $f$ }\}.$$ 
Let $\Size(s) = \cup_{n \in \N} \cup_{m \in \N} \Size_{n,m}(s)$. 
Prove the following:

\textbf{Problem 2.1 (6 points):} $|\Size(s)| =  2^{O(s \log s)}$. 

\textbf{Solution 2.1:}% (write your solution here)

\textbf{Problem 2.2 (12 points):} $|\Size(s)| = 2^{\Omega(s)}$.

\textbf{Solution 2.2:}% (write your solution here)

\textbf{Problem 2.3 (2 points):} $\Size(s) \subsetneq \Size(s^2)$, for every sufficiently large $s$.  

\textbf{Solution 2.3:}% (write your solution here)
}\fi 


\iffalse % Version 1 - Removed because they were covered in 121.5:
\textbf{Problem 2.2 (20 points):}
Let $f,g:\{0,1\}^n \to \{0,1\}$ be two functions that differ one only one input. I.e., there exists $a \in \{0,1\}^n$ such that $f(a) \ne g(a)$ and $f(b) = g(b)$ for all $b \in \{0,1\}^n \setminus \{a\}$. Let $S(f)$ denote the size of the smallest NAND circuit computing $f$ and let $S(g)$ denote the size of the smallest circuit computing $S(g)$. (1) Prove that $S(f) \leq S(g) + 10 n$. 

\textbf{Solution 2.2:} (write your solution here)

\textbf{Problem 2.3 (Bonus, Optional, 0 points):}
Use Problem 2.2 to prove that if $n \leq s \leq 2^n/n - 10n$ then $\Size_{n,1}(s) \subsetneq \Size_{n,1}(s+10n)$. 

\textbf{Solution 2.3:} (write your solution here)
\fi

%{\color{red} In questions below, make sure notation/terminology is consistent with what we use in class. Current notation supposedly based on Barak's book. }

\textbf{Problem 3.1 (15 points):} Describe the set of inputs accepted by the following DFA with 4 states given by $(T,S)$ where $S = \{1,3\}$ and $T$ is given by the following table:
\[
\begin{array}{||c||c|c|c|c||}
\hline
\hline
\mathrm{Input}~\downarrow ~\backslash~ \mathrm{State}~\to & 0 & 1 & 2 & 3 \\
\hline
\hline
0 & 0 & 0 & 1 & 1 \\
\hline
1 & 2 & 2 & 3 & 3 \\
\hline
\hline
\end{array}
\]

\textbf{Solution 3.1:}% (write your solution here)

\textbf{Problem 3.2 (bonus, 0 points):} For all $n$, find, with proof, a regular expression with $10n + 100$ characters that's equivalent to some DFA with $2^n$ states and not equivalent to any DFA with $2^n - 1$ states. (Hint: generalize 3.1.)

\textbf{Solution 3.2:}% (write your solution here)


\textbf{Problem 4 (15 points):} Build a DFA that accepts the regular expression: $(aaa(a|b)^*)|\left((a|b)^*aba(a|b)^*\right)$.

\textbf{Solution 4:}%  (write your solution here)


\textbf{Problem 5 (20 points):} Given a function $f:\{0,1\}^* \to \{0,1\}$, let $f^R:\{0,1\}^* \to \{0,1\}$ be the function given by $f^R(x_1,\ldots,x_n) = 1$ if and only if $f(x_n,\ldots,x_1) = 1$.
Prove that if $f$ is computed by a DFA, then so is $f^R$. 


\textbf{Solution 5:}%  (write your solution here)

\textbf{Problem 6 (20 points):} Prove that the function $f:\{0,1\}^* \to \{0,1\}$ given by $f(x) = 1$ if and only if $x \in \{0^m 1 0^n 1 0^{mn} | m,n \in \N\}$ is not computed by any DFA.

\textbf{Solution 6:}% (write your solution here)



\end{document}
