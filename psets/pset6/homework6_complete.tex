\def\draft{1}

\documentclass[11pt]{article}   
\usepackage{fullpage}
\usepackage{amsfonts}
\usepackage{amssymb}
\usepackage{amsmath}
\usepackage{xcolor}
\usepackage{listings}

\newcommand{\F}{\mathbb{F}}
\newcommand{\np}{\mathop{\rm NP}}
%\newcommand{\binom}[2]{{#1 \choose #2}}
\newcommand{\Z}{{\mathbb Z}}
\newcommand{\vol}{\mathop{\rm Vol}}
\newcommand{\conp}{\mathop{\rm co-NP}}
\newcommand{\atisp}{\mathop{\rm ATISP}}
\renewcommand{\vec}[1]{{\mathbf #1}}
\newcommand{\cupdot}{\mathbin{\mathaccent\cdot\cup}}
\newcommand{\mmod}[1]{\ (\mathrm{mod}\ #1)}  

\setlength{\parskip}{\medskipamount}
\setlength{\parindent}{0in}
%\input{dansmacs}


\begin{document}
	
	\section*{CS 121 Homework 6: Fall
		2020}\label{cs-121-homework-zero-fall-2020}
	
	\textbf{Some policies:} (See the course policy at  {\tt http://madhu.seas.harvard.edu/courses/Fall2020/policy.html} for the
	full policies.)
	
	\begin{itemize}
		\item
		{\bf Collaboration:} You can collaborate with other students that are currently enrolled in
		this course (or, in the case of homework zero, planning to enroll in
		this course) in brainstorming and thinking through approaches to
		solutions but you should write the solutions on your own and cannot
		share them with other students. 
		\item
		{\bf Owning your solution:} Always make sure that you ``own'' your solutions to this other problem
		sets. That is, you should always first grapple with the problems on
		your own, and even if you participate in brainstorming sessions, make
		sure that you completely understand the ideas and details underlying
		the solution. This is in your interest as it ensures you have a solid
		understanding of the course material, and will help in the midterms
		and final. Getting 80\% of the problem
		set questions right on your own will be much better to both your
		understanding than getting 100\% of the questions through
		gathering hints from others without true understanding.
		\item
		{\bf Serious violations:} Sharing questions or solutions with anyone outside this course,
		including posting on outside websites, is a violation of the honor
		code policy. Collaborating with anyone except students currently
		taking this course or using material from past years from this or
		other courses is a violation of the honor code policy.
		\item
		{\bf Submission Format:} The submitted PDF should be typed and in the same format and
		pagination as ours. Please include the text of the problems and write
		\textbf{Solution X:} before your solution. Please mark in gradescope 
		the pages where
		the solution to each question appears. Points will be deducted if you
		submit in a different format.
		\item {\bf Late Day Policy:} To give students some flexibility to manage your schedule, you are allowed a net total of {\bf eight} late days through the semester, but you may not take more than {\bf two} late days on any single problem set. No exceptions to this policy. 
	\end{itemize}
	
	\textbf{By writing my name here I affirm that I am aware of all policies
		and abided by them while working on this problem set:}
	
	\textbf{Your name:} Todd Morrill
	
	\textbf{Collaborators:} (List here names of anyone you discussed
	problems or ideas for solutions with)
	
	% For homeworks 1 through n
	\textbf{No. of late days used on previous psets (not including Homework Zero): } 1\\
	\textbf{No. of late days used after including this pset: } 1
	
	
	\newpage
	

\textbf{Special notes for this homework:}
For your convenience, here's a list of some problems you can assume are in $\mathbf{P}$...
\begin{enumerate}
	\item Linear Programming: given a set of linear inequalities (like $2x_1 - 3x_2 \le 5$), is there a real solution to all of them?
	\item $2$-SAT: given a 2-CNF formula (an AND of clauses each an OR of at most 2 variables or their negations), is there an assignment of variables that makes it true?
	\item $2$-coloring: given a graph $G$, is there a way to color its vertices with 2 colors so that adjacent vertices have different colors?
	\item Shortest Path: given a graph $G$, vertices $s$ and $t$, and an integer $k$, is there a path of length at most $k$ from $s$ to $t$?
	\item Min Cut: given a graph $G$ and an integer $k$, is there a nonempty subset $S \subsetneq V$ with at most $k$ edges from $S$ to $\overline{S}$? 
\end{enumerate}
...and here's a list of some problems you can assume are $\mathbf{NP-complete}$:
\begin{enumerate}
	\item Integer Programming: given a set of linear inequalities (like $2x_1 - 3x_2 \le 5$), is there an integer solution to all of them?
	\item $3$-SAT: given a 3-CNF formula (an AND of clauses each an OR of at most 3 variables or their negations), is there an assignment of variables that makes it true?
	\item $3$-coloring: given a graph $G$, is there a way to color its vertices with 3 colors so that adjacent vertices have different colors?
	\item Longest Path: given a graph $G$, vertices $s$ and $t$, and an integer $k$, is there a path of length at least $k$ from $s$ to $t$? (A path is a sequence of \emph{distinct} vertices with each consecutive pair adjacent.)
	\item Max Cut: given a graph $G$ and an integer $k$, is there a nonempty subset $S \subsetneq V$ with at least $k$ edges from $S$ to $\overline{S}$? 
\end{enumerate}

	\subsection*{Questions}\label{questions}
Please solve the following problems. Some of these might be harder than
the others, so don't despair if they require more time to think or you
can't do them all. Just do your best. Also, you should only attempt the
bonus questions if you have the time to do so. If you don't have a proof
for a certain statement, be upfront about it. You can always explain
clearly what you are able to prove and the point at which you were
stuck. 
%Also, for a non bonus question, 
You can always simply write
\textbf{``I don't know''} and you will get 15 percent of the credit for
the problem. If you are stuck on this problem set, you can use Piazza to
send a private message to all staff.


\textbf{Note on reading the textbook:} If you are stuck on some of the
problems, try consulting the book to 1) understand the concepts the
question is referencing, and 2) review the way similar theorems are
proved in the book.


\newcommand{\HALT}{\mathrm{HALT}}

\textbf{Problem 0 (5 points):} True or False: I have completed the midterm 2 feedback survey. (True worth 5 points, False, or I don't know worth 0 points.)

\textbf{Solution 0:} True

\newpage

\textbf{Problem 1.1 (10 points):} Suppose that you are in charge of
scheduling courses in computer science in University S. In University S,
computer science students wake up late, and have to work on their
startups in the afternoon, and take long weekends with their investors.
So you only have two possible slots: you can schedule a course either
Monday-Wednesday 11am-1pm or Tuesday-Thursday 11am-1pm.

Let \(SCHEDULE_S:\{0,1\}^* \rightarrow \{0,1\}\) be the function that
takes as input a list of courses \(L\) and a list of \emph{conflicts}
\(C\) (i.e., list of pairs of courses that cannot share the same time
slot) and outputs \(1\) if and only if there is a ``conflict free''
scheduling of the courses in \(L\) to the two slots, where no pair in
\(C\) is scheduled in the same time slot. (We model the course list
\(L\) as just a list of strings, with the conflict lists \(C\) as a list
of pairs of strings.)

Prove that \(SCHEDULE_S \in \mathbf{P}\). As usual, you do not have to
provide the full code to show that this is the case, and can describe
operations as a high level, as well as appeal to any data structures or
other results mentioned in the book or in lecture. Note that to show
that a function \(F\) is in \(\mathbf{P}\) you need to \textbf{(1)}
present an algorithm \(A\), 
\textbf{(2)} prove that \(A\) computes \(F\) in polynomial time, and
\textbf{(3)} prove that \(A\) runs in polynomial time. See footnote for hint.\footnote{Try to see how you model the setting
	mathematically, and whether it amounts to questions about objects we
	have looked at before.}

\textbf{Solution 1.1:}

We can define an algorithm $A$ that computes $SCHEDULE_S$ because it can be reduced to the decision version of the graph 2-coloring problem. We let $G = (L, C)$, where the list of courses $L$ corresponds to the vertices of the graph and the conflicts $C$ correspond to the edges of the graph. This transformation can be done in time $O(n + m)$ where $n$ is the number of courses in $L$ and $m$ is the number of conflicts in $C$. 

Once the input to $A$ is converted to $G$, we can simply call the procedure which computes the 2-coloring function, which we know is in $\mathbf{P}$ as specified in the instructions to the problem set. NB: if a course (i.e. a node) has no conflicts (i.e. no edges) then it can be arbitrarily assigned to the schedule (i.e. a color).

We now show that $SCHEDULE_S(L, C) = 1$ iff $A(L, C) = 1$.

\textbf{Completeness:} If there exists a conflict-free schedule then the corresponding graph is 2-colorable. Course conflicts correspond to edges in the graph. For any arbitrary conflict, one course will be assigned Monday/Wednesday and the other will be assigned Tuesday/Thursday. But this means that nodes on either side of the corresponding edge can be assigned different colors such as red and blue.

\textbf{Soundness:} If the graph is 2-colorable then there exists a conflict-free schedule. Since the graph is 2-colored, nodes on each end of the edge will have different colors. These edges correspond to course conflicts. Since the nodes have different colors, the corresponding courses can be assigned to different days to create a conflict-free schedule.

Since the transformation step and the 2-coloring procedure are both computable in polynomial time and algorithm $A$ correctly computes $SCHEDULE_S$, we conclude that $SCHEDULE_S \in \mathbf{P}$.

\newpage

\textbf{Problem 1.2 (10 points):} Consider the same question but now at
university H where there is a third course slot on Monday-Wednesday from
11pm till 1am. Let \(SCHEDULE_H:\{0,1\}^* \rightarrow \{0,1\}\) be the
function as above that takes as input a list of courses \(L\) and a list
of \emph{conflicts} \(C\) and outputs \(1\) if and only if there is a
``conflict free'' scheduling of the courses in \(L\) to the three slots
of \(H\). Prove that \(SCHEDULE_H\) is \(\mathbf{NP}\)-complete.

\textbf{Solution 1.2:}

To show that $SCHEDULE_H$ is \(\mathbf{NP}\)-complete, we must show that $SCHEDULE_H \in \mathbf{NP}$ and $SCHEDULE_H \in \mathbf{NP}$-hard.

$\underline{SCHEDULE_H \in \mathbf{NP}}$

Let $V: \{0, 1\}^* \to \{0, 1\}$ be a verification function that accepts $\rho = (L, C)$, which is an instance of the input to $SCHEDULE_H$ and $x$, which is a set of 3 sets - 1 for each scheduling slot, where each $l \in L$ is assigned to exactly 1 of the 3 sets in $x$. $V(\rho, x) = 1$ iff $SCHEDULE_H(L, C) = 1$. The procedure to compute $V$ is as follows.

\begin{lstlisting}[language=Python]
def compute_v(rho, x):
    # check that all courses have been scheduled
    for l in L:
        if l not in x[0], x[1], or x[2]:
            return 0
    
    # check that all conflicts are satisfied
    for a, b in C:
        for set in x:
            if a in set and b in set:
                return 0
    
    # passed verification check
    return 1
\end{lstlisting}

compute\_v runs in time $O(|L| + |C|)$, which is polynomial. From this we conclude that $SCHEDULE_H \in \mathbf{NP}$.

$\underline{SCHEDULE_H \in \mathbf{NP}}$-hard: $3-COLORING \leq_p SCHEDULE_H$

The decision version of the function $3-COLORING$ accepts a graph $G = (V, E)$ and returns 1 iff the graph's vertices can be colored using at most 3 colors such that no two neighboring vertices are the same color. The problem set instructions tell us that $3-COLORING$ is \(\mathbf{NP}\)-complete, which tells us that $3-COLORING$ is $\mathbf{NP}$-hard. Showing that $3-COLORING$ can be reduced to $SCHEDULE_H$ will allow us to conclude that $SCHEDULE_H$ is $\mathbf{NP}$-hard.

$3-COLORING$ can be reduced to $SCHEDULE_H$. In particular, the set of vertices $V$ can be mapped to courses $L$ and the set of edges $E$ can be mapped to course conflicts $C$. The 3 node colors correspond to the 3 time slots for courses.

This translation, which we denote as $R$, from an input to $3-COLORING$ to an input to $SCHEDULE_H$ takes $O(|V| + |E|)$ time, which is polynomial time.

We now prove that $3-COLORING(G) = SCHEDULE_H(R(G))$.

\textbf{Completeness:} if the graph $G$ is 3-colorable, then there exists a conflict-free schedule with 3 time slots. Since the graph is 3-colored, all neighboring nodes are assigned different colors. Since edges correspond to conflicts and nodes correspond to courses, each course will be scheduled to a time slot according to its node's color and all conflicts will be satisfied. 

\textbf{Soundness:} if there exists a conflict-free schedule with 3 time slots, then the graph is 3-colorable. Since the schedule is conflict-free, all conflicts in $C$ are satisfied and each course in the pair is scheduled to a different time slot. Since each time slot corresponds to a color and conflicts correspond to edges in the graph, each neighboring node will be assigned a different color and will satisfy the 3-coloring condition.

From this we conclude that $SCHEDULE_H \in \mathbf{NP}$-hard. This shows that \(SCHEDULE_H\) is \(\mathbf{NP}\)-complete, which concludes the proof.

\newpage

\textbf{Problem 2} Recall that $\mathbf{P}$, $\mathbf{NP}$, and other complexity classes are sets of \emph{Boolean} functions $f: \{0,1\}^* \to \{0,1\}$; for instance, 3SAT returns 1 iff there exists a satisfying assignment to its input formula. In this problem, we want a polynomial-time algorithm to \emph{find} a satisfying assignment.

\textbf{Problem 2.1 (5 points)} If $\phi(x_0, x_1, \ldots, x_{n-1})$ is a 3-CNF formula and $i \in \{0,1\}$, let $\phi_i(x_1, \ldots, x_{n-1}) = \phi(i,x_1, \ldots, x_{n-1})$. Show that, given $3SAT(\phi_0)$ and $3SAT(\phi_1)$, you can calculate $3SAT(\phi)$.

\textbf{Solution 2.1}

There are 2 mutually exclusive and collectively exhaustive cases that allow us to calculate $3SAT(\phi)$.

\textbf{Case 1:} (($3SAT(\phi_0) = 1) \lor (3SAT(\phi_1) = 1)) \implies 3SAT(\phi) = 1$

\textbf{Case 2:} (($3SAT(\phi_0) = 0) \land (3SAT(\phi_1) = 0)) \implies 3SAT(\phi) = 0$

This shows that given $3SAT(\phi_0)$ and $3SAT(\phi_1)$, we can calculate $3SAT(\phi)$.

\newpage


\textbf{Problem 2.2 (15 points)} Suppose you have a polynomial-time algorithm $M$ that computes \textsc{3SAT}. Give a polynomial-time algorithm $M'$ that, on input a 3-CNF formula $\phi(x_0, x_1, \ldots, x_{n-1})$, returns a satisfying assignment if one exists, or 0 if none does.

\textbf{Solution 2.2}

Building off of \textbf{solution 2.1}, we can hardcode partial satisfying assignments to build up the correct solution one bit at a time. In particular, we want to build up a satisfying solution $x$ such that $\phi(x) = 1$. The approach is as follows.

\begin{lstlisting}[language=Python]
def embed_z(phi, z):
    """Check if z is a valid starting sequence for the solution x"""
    phi_z = phi with the first |z| bits of solution x set to z
    return M(phi_z)
\end{lstlisting}

\begin{lstlisting}[language=Python]
def M_prime(phi):
    """Find solution x"""
    # build up solution 1 bit at a time
    z = ''
    n = number of variables in phi expression
    for _ in len(n):
        solution_0 = embed_z(phi, z + '0')
        solution_1 = embed_z(phi, z + '1')
        if solution_0 == solution_1 == 0:
            return 0
        elif solution_0 == 1:
            z = z + '0'
        else:
            z = z + '1'
    return z
\end{lstlisting}

\textbf{Correctness:} As demonstrated, we test partial solutions to $x$ one bit at a time, checking if 0 or 1 can be appended to the partial solution $z$. We hard code $\phi$ with the first $|z|$ bits of $x$ set to $z$. We use $M$ to determine if adding a 0 or 1 to this partial solution $z$ will result in a valid partial solution. We continue in this manner for $n$ steps where $n = |x|$. If the loop finishes, then $M'$ returns a valid solution for $x$. If $(M(\phi_0) = 0) \land (M(\phi_1) = 0)$ on the first iteration of the loop, then $M'$ returns 0 because no satisfying assignment exists.

\textbf{Running time:} This solution has one main loop that executes $n = |x|$ times and on each iteration of the loop calls $M$, which is given to be a polynomial-time algorithm. Together this yields a polynomial-time algorithm $M'$.

This shows that there exists a polynomial-time algorithm $M'$ that, on input a 3-CNF formula $\phi(x_0, x_1, \ldots, x_{n-1})$, returns a satisfying assignment if one exists, or 0 if none does.

\newpage

\textbf{Problem 2.3 (Bonus, 0 points (hard))} Give an algorithm $M'$ with the following property: If it is true that $\textsc{3SAT} \in P$, then, on input a satisfiable 3-CNF formula $\phi(x_0, x_1, \ldots, x_{n-1})$, $M'$ returns a satisfying assignment in polynomial time. (If no satisfying assignment to its input formula exists, $M'$ may not halt in polynomial time.)

\newpage

\textbf{Problem 2.4 (25 points)} Suppose you have a polynomial-time algorithm $M$ that computes Longest Path. Give a polynomial-time algorithm $M'$ that, on input a graph $G$, vertices $s$ and $t$, and an integer $k$, returns a path of length at least $k$ from $s$ to $t$ if one exists, or 0 if none does.

\textbf{Solution 2.4}

The proof closely follows the pattern of \textbf{solution 2.2}, by building up a satisfying path 1 node at a time. The algorithm $M'$ is as follows.


\begin{lstlisting}[language=Python]
def M_prime(G, s, t, k):
    """
    Find path between s and t in graph G with length greater than or equal to k
    """
    # check if there exists a solution
    if not M(G, s, t, k):
        return 0
    
    # build up solution 1 node at a time
    k_working = k
    path = [s]
    while s != t:
        for u in neighbors of s:
            k_temp = max(k_working - 1, 0)
            if M(G, u, t, k_temp):
                path.append(u)
                k_working = k_temp
                s = u
                break
    return path
\end{lstlisting}

\textbf{Correctness:} The correctness of $M'$ heavily relies on the correctness of $M$. Since we assume $M$ to be correct, we can quickly determine if there exists a solution by simply checking if $M(G, s, t, k) = 1$. If $M(G, s, t, k) = 0$, then no such path exists and we simply return 0 as specified in the problem statement. If a path satisfying the inputs does exist, we build up a solution 1 node at a time, relying on the correctness of $M$ to determine if a particular node is a valid addition to the path. At each iteration of the while loop, we reduce the the number $k$ by 1 to reflect the fact that we are 1 node closer to finding a valid path. The path may be longer than $k$ so we prevent $k$ from being negative with max(k\_working - 1, 0). The loop terminates when $s = t$ (i.e. when a complete path has been found) and $M'$ returns a satisfying path.

\textbf{Running time:} This solution has one main while loop that executes up to $|E|$ times, where $E$ is the set of edges in the graph, and one nested for loop that executes up to $|E|$ times. This results in $O(|E|^2)$ invocations of $M$, which is given to be a polynomial-time algorithm. Together this yields a polynomial-time algorithm $M'$.

This shows that there exists a polynomial-time algorithm $M'$ that, on input a graph $G$, vertices $s$ and $t$, and an integer $k$, returns a path of length at least $k$ from $s$ to $t$ if one exists, or 0 if none does.

\newpage

\textbf{Question 3.1 (15 points):} \textsc{$k$-1s-Positive-3SAT} is the following
function: the input is a CNF formula \(\varphi\) where each clause is
the OR of one to three variables (\emph{without negations}), and a
number \(k\in \mathbb{N}\). For example, the following is a valid input:
\(\left(\varphi = (x_5 \vee x_{2} \vee x_1) \wedge (x_1 \vee x_3) \wedge (x_2 \vee x_4 \vee x_0), k = 2\right)\).
The output is 1 if and only if there exists a
satisfying assignment to \(\varphi\) in which exactly \(k\) of the
variables get the value \(1\). For example, for the formula \(\varphi\)
above, \(\textsc{$k$-1s-Positive-3SAT}(\varphi,2)=1\) since the assignment \((1,1,0,0,0,0)\)
satisfies all the clauses. However \(\textsc{$k$-1s-Positive-3SAT}(\varphi,1)=0\) since there is
no single variable appearing in all clauses.

Prove that \textsc{$k$-1s-Positive-3SAT} is \(\mathbf{NP}\)-complete. See footnote for
hint.\footnote{\textbf{Hint:} You can use a direct reduction from
	\(3SAT\) or a reduction from a function that was shown to be
	\(\mathbf{NP}\)-complete in either the textbook, lecture, or section.
	If you go via the direct reduction route, you might want to try to
	transform a SAT instance on \(n\) variables \(x_0,\ldots,x_{n-1}\) to
	a \textsc{$k$-1s-Positive-3SAT} instance on \(2n\) variables \(w_0,\ldots,w_{2n}\) so that for
	every \(i\in [n]\), the variable \(w_{2i}\) will correspond to \(x_i\)
	and the variable \(w_{2i+1}\) will correspond to \(\neg x_i\). You can
	try to find ways to constrain a weight \(n\) solution so that exactly
	one of the variables \(w_{2i}\) and \(w_{2i+1}\) will equal \(1\) for
	every \(i \in [n]\).}

\textbf{Solution 3.1:}
To show that \textsc{$k$-1s-Positive-3SAT} is \(\mathbf{NP}\)-complete, we must show that \textsc{$k$-1s-Positive-3SAT} $\in \mathbf{NP}$ and \textsc{$k$-1s-Positive-3SAT} $\in \mathbf{NP}$-hard.

\textsc{$k$-1s-Positive-3SAT} $\in \mathbf{NP}$

Let $V: \{0, 1\}^* \to \{0, 1\}$ be a verification function that accepts $\rho = (\varphi, k)$, which consists of a CNF formula $\varphi$, $k\in \mathbb{N}$, and an assignment $x$ to the variables in $\varphi$. $V(\rho, x) = 1$ iff \textsc{$k$-1s-Positive-3SAT}($\varphi$, $k$) = 1. The procedure to compute $V$ is as follows.

\begin{lstlisting}[language=Python]
def compute_v(rho, x):
    phi, k = rho
    
    # check that x contains exactly k 1s
    if k != sum(x):
        return 0
    
    for clause in phi:
        clause_sum = 0
        for literal in clause:
            # retrieve the assignment to the literal
            temp = x[literal]
            clause_sum += temp
        
        # check if the clause was satisfied
        if clause_sum == 0:
            return 0
    
    # passed verification check
    return 1
\end{lstlisting}

compute\_v runs in time $O(m)$, where $m$ is the number of clauses in $\varphi$, which runs in polynomial-time. From this we conclude that \textsc{$k$-1s-Positive-3SAT} $\in \mathbf{NP}$.\\

\textsc{$k$-1s-Positive-3SAT} $\in \mathbf{NP}$-hard: $3SAT \leq_p$ \textsc{$k$-1s-Positive-3SAT}

We know that $3SAT$ is $\mathbf{NP}$-hard. Showing that $3SAT$ can be reduced to \textsc{$k$-1s-Positive-3SAT} will allow us to conclude that \textsc{$k$-1s-Positive-3SAT} is $\mathbf{NP}$-hard.

Variables $x_0 \cdots x_{n-1}$ in the original 3SAT instance correspond to variables $w_0 \cdots w_{2n-1}$ in the \textsc{$k$-1s-Positive-3SAT} setting, where $x_i$ corresponds to $w_{2i}$ and $\overline{x}_i$ corresponds to $w_{2i + 1}$, for all $n$ variables.

For the reduction, we simply translate $x_i$ to $w_{2i}$ or $\overline{x}_i$ to $w_{2i+1}$ in all the clauses found in original $3SAT$ instance. We set $k=n=|x|$ because a satisfying solution in the \textsc{$k$-1s-Positive-3SAT} setting should contain exactly $n$ 1s. To enforce the constraint that exactly 1 of each pair $w_{2i}, w_{2i+1}$ must equal 1, we add $n$ additional clauses of form $(w_{2i} \lor w_{2i+1})$ for all $i \in n$.

This reduction $R$ from an input to $3SAT$ to an input to \textsc{$k$-1s-Positive-3SAT} takes $O(m + n)$ time, where we must replace the variables in the $m$ $3SAT$ clauses and add an additional $n$ clauses of the form $(w_{2i} \lor w_{2i+1})$. This reduction runs in polynomial time.

We now prove that $3SAT(\varphi) = $\textsc{$k$-1s-Positive-3SAT}$(R(\varphi))$.

\textbf{Completeness:} If $3SAT(\varphi) = 1$, then \textsc{$k$-1s-Positive-3SAT}$(R(\varphi)) = 1$. If there exists a satisfying assignment for $\varphi$ then each clause must have been satisfied and so by simple translation of variables the first $m$ clauses of $(R(\varphi))$ will be satisfied. Since $x_i$ in $\varphi$ is either 0 or 1 (and $\overline{x}$ takes the opposite value) exactly 1 of $w_{2i}, w_{2i+1}$ will equal 1 for all $i \in n$, hence the choice of $k=n=|x|$. This also satisfies the $n$ remaining clauses of the form $(w_{2i} \lor w_{2i+1})$.

\textbf{Soundness:} If \textsc{$k$-1s-Positive-3SAT}$(R(\varphi)) = 1$, then $3SAT(\varphi) = 1$. We set $k=n$ where $n$ is the number of variables in $\varphi$. The additional $n$ clauses of the form $(w_{2i} \lor w_{2i+1})$ constrain solutions to those where exactly 1 of $w_{2i}, w_{2i+1}$ equal 1 for all $i \in n$, hence, there will be no contradictions such as $x_i = \overline{x}_i = 1$ when translating $R(\varphi)$ back to $\varphi$. Since the first $m$ clauses of $R(\varphi)$ are satisfiable, a simple translation of the $w$ variables back $x$ variables will satisfy the $m$ clauses of $\varphi$. 

From this we conclude that \textsc{$k$-1s-Positive-3SAT}$\in \mathbf{NP}$-hard. This shows that \textsc{$k$-1s-Positive-3SAT} is \(\mathbf{NP}\)-complete, which concludes the proof.

\newpage

\textbf{Question 3.2 (15 points):} In the \emph{employee recruiting
	problem} we are given a list of potential employees, each of which has
some subset of \(m\) potential skills, and a number \(k\). We want a team of \(k\) employees such that for every skill there is at least one member of the team with it.

For example, if Alice has the skills ``C programming'', ``NAND
programming'' and ``Solving Differential Equations'', Bob has the skills
``C programming'' and ``Solving Differential Equations'', and Charlie
has the skills ``NAND programming'' and ``Coffee Brewing'', then if we
want a team of $k=2$ people that covers all $n=4$ skills, we could hire
Alice and Charlie.

Define the function \(EMP\) s.t. on input the skills \(L\) of all
potential employees (in the form of a sequence \(L\) of \(n\) lists
\(L_1,\ldots,L_n\), each containing distinct numbers between \(0\) and
\(m\)), and a number \(k\), \(EMP(L,k)=1\) if and only if there is a
subset \(S\) of \(k\) potential employees such that for every skill
\(j\) in \([m]\), there is at least one employee in \(S\) that has the
skill \(j\).

Prove that \(EMP\) is \(\mathbf{NP}\)-complete. You can use the result
of the previous subquestion even if you didn't prove it. See also
footnote.\footnote{You can assume that lists (some of which could
	potentially be empty) are represented as strings in some reasonable
	way, and for every skill \(j\) there is some employee with the skill
	\(j\), and so in particular both \(n\) and \(m\) are smaller than the
	length of the input as a string of bits.}

\textbf{Solution 3.2:}
To show that \(EMP\) is \(\mathbf{NP}\)-complete, we must show that $EMP \in \mathbf{NP}$ and $EMP \in \mathbf{NP}$-hard.

$EMP \in \mathbf{NP}$

Let $V: \{0, 1\}^* \to \{0, 1\}$ be a verification function that accepts $\rho = (L, k)$, which consists of a sequence \(L\) of \(n\) lists \(L_1,\ldots,L_n\) each containing distinct numbers between \(0\) and \(m\), and $k\in \mathbb{N}$, and a subset $S$ of $k$ employees. $V(\rho, S) = 1$ iff $EMP(L, k$) = 1. The procedure to compute $V$ is as follows.

\begin{lstlisting}[language=Python]
def compute_v(rho, S):
    L, k = rho
    
    # check that S contains exactly k employees
    if k != |S|:
        return 0
    
    # compute complete skill set
    complete_skill_set = {}
    for skill_set in L:
        complete_skill_set = complete_skill_set.union(skill_set)
    
    # compute employee skill set
    employee_skill_set = {}
    for employee in S:
        # retrieve skills for employee
        temp = L[employee]
        employee_skill_set = employee_skill_set.union(temp)
        
    # check that all skills are covered
    if complete_skill_set != employee_skill_set:
        return 0
    
    # passed verification check
    return 1
\end{lstlisting}

compute\_v runs in time $O(n + k) = O(n)$, where $n$ is the number of candidate employees, $k$ is the number of employees selected and $k \leq n$. This runs in polynomial-time. From this we conclude that \(EMP\) $\in \mathbf{NP}$.\\

$EMP \in \mathbf{NP}$-hard: \textsc{$k$-1s-Positive-3SAT} $\leq_p EMP$ 

We know that \textsc{$k$-1s-Positive-3SAT} is $\mathbf{NP}$-hard by \textbf{solution 3.1}. Showing that \textsc{$k$-1s-Positive-3SAT} can be reduced to $EMP$ will allow us to conclude that $EMP$ is $\mathbf{NP}$-hard.

Variables $x_0 \cdots x_{n-1}$ in the original \textsc{$k$-1s-Positive-3SAT} instance correspond to the $n$ candidate employees in the $EMP$ setting, and the $m$ clauses in \textsc{$k$-1s-Positive-3SAT} correspond to the $m$ skills employees may have. In particular, we number the CNF clauses $0 \cdots m$ and if an employee (i.e. $x_i$) is a member of CNF clause $m_j$, then they have skill $m_j$. The $k$ ones in satisfying assignments to CNF formula $\varphi$ correspond to the $k$ employees needed to cover all the $m$ skills.

For the reduction, we simply assign a number to each of the $m$ clauses and then for each variable $x_i$ in the \textsc{$k$-1s-Positive-3SAT} we create a list $L_i$ that corresponds to the skills that employee $i$ has, where the skills correspond to the clauses that employee $i$ was a part of. $k$ in the original problem simply maps to $k$ in the reduced problem.

This reduction $R$ from an input to \textsc{$k$-1s-Positive-3SAT} to an input to $EMP$ takes $O(m + n)$ time, where we must iterate over the $m$ CNF clauses and create $n$ employee skills lists. This reduction runs in polynomial time.

We now prove that \textsc{$k$-1s-Positive-3SAT}$(\varphi) = EMP(R(\varphi))$.

\textbf{Completeness:} If \textsc{$k$-1s-Positive-3SAT}$(\varphi, k) = 1$, then $EMP(R(\varphi, k)) = 1$. If there exists a satisfying assignment $x$ for $\varphi$ then each clause must have been satisfied by the $k$ ones in $x$. This means that there are $k$ employees that collectively possess all the $m$ skills.

\textbf{Soundness:} If $EMP(R(\varphi, k)) = 1$, then \textsc{$k$-1s-Positive-3SAT}$(\varphi, k) = 1$. We select $k$ $L_i$ employee skill lists that correspond to the $k$ $x_i$ variables that are set to 1 in the satisfying solution $x$. Since these $k$ employees collectively possess all $m$ skills, this means the $k$ variables $x_i$ satisfy all $m$ clauses of $\varphi$. 

From this we conclude that $EMP \in \mathbf{NP}$-hard. This shows that $EMP$ is \(\mathbf{NP}\)-complete, which concludes the proof.

\newpage

\textbf{Problem 4 (20 points):} Prove that the ``balanced variant'' \emph{BMC} of
the maximum cut problem is \(\mathbf{NP}\)-complete: given a graph $G$ and an integer $k$, $BMC(G,k) = 1$ iff there is a subset $S \subsetneq V$ of size $|V|/2$ with at least $k$ edges from $S$ to $\overline{S}$.

\textbf{Solution 4:}

To show that $BMC$ is \(\mathbf{NP}\)-complete, we must show that $BMC \in \mathbf{NP}$ and $BMC \in \mathbf{NP}$-hard.

$BMC$ \in \mathbf{NP}$

Let $V': \{0, 1\}^* \to \{0, 1\}$ be a verification function that accepts $\rho = (G, k)$, which consists of a graph $G$ and $k\in \mathbb{N}$, and a subset $S$ of  $V$. $V'(\rho, S) = 1$ iff $BMC(G, k$) = 1. The procedure to compute $V'$ is as follows.

\begin{lstlisting}[language=Python]
def compute_v_prime(rho, S):
    G, k = rho
    V, E = G
    
    # check that S contains exactly |V|/2 nodes
    if |S| != |V|/2:
        return 0
    
    # check that there are at least k edges crossing the cut
    k_check = 0
    for edge in E:
        u, v = edge
        if (u in S) and (v in V-S):
            k_check += 1
        elif (v in S) and (u in V-S):
            k_check += 1

    if k_check < k:
        return 0
    
    # passed verification check
    return 1
\end{lstlisting}

compute\_v\_prime runs in time $O(|E|)$ to check that there are at least $k$ edges crossing the cut, which is polynomial-time. From this we conclude that \(BMC\) $\in \mathbf{NP}$.\\

$BMC \in \mathbf{NP}$-hard: $MAXCUT \leq_p BMC$ 

The problem set instructions tell us that $MAXCUT$ is $\mathbf{NP}$-hard. Showing that $MAXCUT$ can be reduced to $BMC$ will allow us to conclude that $BMC$ is $\mathbf{NP}$-hard.

For the reduction we can add $|V|$ nodes to the original graph $G$ to achieve a balanced max cut. This results in a new graph $G'$ with $|V'| = 2|V|$ nodes. No edges are modified. $k$ in the original problem simply maps to $k$ in the reduced problem.

This reduction $R$ from an input to $MAXCUT$ to an input to $BMC$ takes $O(|V|)$ time to simply add $|V|$ nodes. This reduction runs in polynomial time.

We now prove that $MAXCUT(G, k) = BMC(R(G, k))$.

\textbf{Completeness:} If $MAXCUT(G, k) = 1$, then $BMC(R(G, k)) = 1$. If there exists a satisfying subset $S$ then this means there are at least $k$ edges crossing the cut in $G$. In particular, $|V| - |S|$ nodes are added to $S$ to create $S'$. The remainder of the $|V|$ nodes are added to $\overline{S}$. This guarantees that $|S'| = |V'| / 2$, where $S'$ is a satisfying subset in the $BMC$ setting. Since the set of edges is unchanged, the number of edges crossing the cut is still at least $k$.

\textbf{Soundness:} If $BMC(R(G, k)) = 1$, then $MAXCUT(G, k) = 1$. Since $S'$ is a satisfying subset for $BMC$ with at least $k$ edges crossing the cut, we can simply remove the $|V'|/2$ nodes that were added to achieve a subset $S$. Since no edges were modified, this subset $S$ will have at least $k$ edges crossing the cut in the $MAXCUT$ setting.

From this we conclude that $BMC \in \mathbf{NP}$-hard. This shows that $BMC$ is \(\mathbf{NP}\)-complete, which concludes the proof.

\newpage

\newcommand{\CSAT}{\mathrm{CubicSAT}}

\textbf{Problem 5.1 (20 points):} The input to the $\CSAT$ function is $m$ polynomials $P_0,\ldots,P_{m-1}$ on $n$ variables $x_0,\ldots,x_{n-1}$. Each polynomial is a cubic equation in the variables, i.e., the degree of every monomial is at most $3$. Also, the coefficients are integers in the range $\{-n,\ldots,n\}$. (Each such polynomial can be expressed as a list of $O(n^3)$ integers - why?). $\CSAT(P_0,\ldots,P_{m-1}) = 1$ iff there exist integers $a_0,\ldots,a_{n-1}$ such that for every $j \in [m]$, $P_j(a_0,\ldots,a_{n-1})=0$. Prove that $\CSAT$ is NP-hard.

\textbf{Solution 5.1}

To show that $\CSAT$ is \(\mathbf{NP}\)-complete, we must show that $\CSAT \in \mathbf{NP}$ and $\CSAT \in \mathbf{NP}$-hard.

$\CSAT \in \mathbf{NP}$

Let $V: \{0, 1\}^* \to \{0, 1\}$ be a verification function that accepts $\rho = P_0,\ldots,P_{m-1}$, which consists of $m$ polynomials on $n$ variables, and a sequence of integers $a = a_0,\ldots,a_{n-1}$. $V(\rho, a) = 1$ iff $\CSAT(P_0,\ldots,P_{m-1}) = 1$. The procedure to compute $V$ is as follows.

\begin{lstlisting}[language=Python]
def compute_v(rho, a):
    n = number of unique vars in rho
    
    # check that a has length n
    if len(a) != n:
        return 0
    
    # check that each polynomial is satisfied
    for p in rho:
        if p(a) != 0:
            return 0

    # passed verification check
    return 1
\end{lstlisting}

compute\_v runs in time $O(m)$, where $m$ is the number of polynomials. This runs in polynomial-time. From this we conclude that $\CSAT \in \mathbf{NP}$.\\

$\CSAT \in \mathbf{NP}$-hard: $3SAT \leq_p \CSAT$ 

We know that $3SAT$ is $\mathbf{NP}$-hard. Showing that $3SAT$ can be reduced to $\CSAT$ will allow us to conclude that $\CSAT$ is $\mathbf{NP}$-hard.

For the reduction we can translate CNF clauses to polynomials. There are $n$ variables $y_0 \ldots y_{n-1}$ in the $3SAT$ setting, which correspond to the $n$ variables $x_0 \ldots x_{n-1}$ in the $\CSAT$ setting. Similarly, the $m$ clauses in the CNF formula $\varphi$ are mapped to the $m$ polynomials. Each positive literal $y_i$ in clause $m_j$ in the $3SAT$ setting will be mapped to $(x_i - 1)$ in the $m_j^{th}$ polynomial in the $\CSAT$ setting. Similarly, negative literals $\overline{y}_i$ in clause $m_j$ will be mapped to $x_i$ in the $m_j^{th}$ polynomial in the $\CSAT$ setting. An example translation looks like the following: $(y_1 \lor y_1 \lor \overline{y}_3) \to (x_1 - 1)(x_1 - 1)(x_3) = x_3x_1^2 - 2x_3x_1 - x_3$

This reduction $R$ from an input to $3SAT$ to an input to $\CSAT$ takes $O(m)$ time, where $m$ is the number clauses in $\varphi$. This reduction runs in polynomial time.

We now prove that $3SAT(\varphi) = \CSAT(R(\varphi))$.

\textbf{Completeness:} If $3SAT(\varphi) = 1$, then $\CSAT(R(\varphi)) = 1$. Since $y$ is a satsifying assignment to $\varphi$, this means that every clause $m_j$ had at least one literal that equals 1. This means that at least one of the roots of the polynomial $m_j$ will be zero. For example, if $y_i = 1$, then $x_i = 1$ and in turn $(x_i - 1) = 0$. This construction means that all $m$ polynomials will evaluate to 0. The final checks are 1) for the constraint that the resulting polynomials are cubic equations and 2) for the constraint that coefficients of the polynomials are in the range $\{-n, \ldots, n\}$. For all $n\geq 3$, this will be the case. CNF clauses of the form $(y_i \lor y_i \lor y_i)$ yield coefficients and degrees of the greatest magnitude (i.e. $ (x_i - 1)(x_i - 1)(x_i - 1) = x^3 - 3x^2 - 3x - 1$). This shows that that all monomials will have degree and coefficients at most 3. There are only a small number of $3SAT$ instances with non-redundant clauses with $n < 3$ where this does not hold.

\textbf{Soundness:} If $\CSAT(R(\varphi)) = 1$, then $3SAT(\varphi) = 1$. Since there exists a solution $a$ on $n$ variables that satisfies all the $m$ cubic polynomial equations, we know that we can unpack the roots of these equations into $m$ CNF clauses (max 3 variables per clause because equations are cubic) on $n$ variables. In particular, we map roots of the form $(x_i - 1)$ in the $m_j^{th}$ polynomial to $y_i$ in the $m_j^{th}$ clause of $\varphi$ and similarly, we map $(x_i)$ to $\overline{y}_i$. This shows that all $m$ clauses of $\varphi$ will be satisfied.

From this we conclude that $\CSAT \in \mathbf{NP}$-hard. This shows that $\CSAT$ is \(\mathbf{NP}\)-complete, which concludes the proof.

\end{document}
