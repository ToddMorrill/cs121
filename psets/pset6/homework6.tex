\def\draft{1}

\documentclass[11pt]{article}   
\usepackage{fullpage}
\usepackage{amsfonts}
\usepackage{amssymb}
\usepackage{amsmath}
\usepackage{xcolor}

\newcommand{\F}{\mathbb{F}}
\newcommand{\np}{\mathop{\rm NP}}
%\newcommand{\binom}[2]{{#1 \choose #2}}
\newcommand{\Z}{{\mathbb Z}}
\newcommand{\vol}{\mathop{\rm Vol}}
\newcommand{\conp}{\mathop{\rm co-NP}}
\newcommand{\atisp}{\mathop{\rm ATISP}}
\renewcommand{\vec}[1]{{\mathbf #1}}
\newcommand{\cupdot}{\mathbin{\mathaccent\cdot\cup}}
\newcommand{\mmod}[1]{\ (\mathrm{mod}\ #1)}  

\setlength{\parskip}{\medskipamount}
\setlength{\parindent}{0in}
%\input{dansmacs}


\begin{document}
	
	\section*{CS 121 Homework 6: Fall
		2020}\label{cs-121-homework-zero-fall-2020}
	
	\textbf{Some policies:} (See the course policy at  {\tt http://madhu.seas.harvard.edu/courses/Fall2020/policy.html} for the
	full policies.)
	
	\begin{itemize}
		\item
		{\bf Collaboration:} You can collaborate with other students that are currently enrolled in
		this course (or, in the case of homework zero, planning to enroll in
		this course) in brainstorming and thinking through approaches to
		solutions but you should write the solutions on your own and cannot
		share them with other students. 
		\item
		{\bf Owning your solution:} Always make sure that you ``own'' your solutions to this other problem
		sets. That is, you should always first grapple with the problems on
		your own, and even if you participate in brainstorming sessions, make
		sure that you completely understand the ideas and details underlying
		the solution. This is in your interest as it ensures you have a solid
		understanding of the course material, and will help in the midterms
		and final. Getting 80\% of the problem
		set questions right on your own will be much better to both your
		understanding than getting 100\% of the questions through
		gathering hints from others without true understanding.
		\item
		{\bf Serious violations:} Sharing questions or solutions with anyone outside this course,
		including posting on outside websites, is a violation of the honor
		code policy. Collaborating with anyone except students currently
		taking this course or using material from past years from this or
		other courses is a violation of the honor code policy.
		\item
		{\bf Submission Format:} The submitted PDF should be typed and in the same format and
		pagination as ours. Please include the text of the problems and write
		\textbf{Solution X:} before your solution. Please mark in gradescope 
		the pages where
		the solution to each question appears. Points will be deducted if you
		submit in a different format.
		\item {\bf Late Day Policy:} To give students some flexibility to manage your schedule, you are allowed a net total of {\bf eight} late days through the semester, but you may not take more than {\bf two} late days on any single problem set. No exceptions to this policy. 
	\end{itemize}
	
	\textbf{By writing my name here I affirm that I am aware of all policies
		and abided by them while working on this problem set:}
	
	\textbf{Your name:} (Write name and HUID here)
	
	\textbf{Collaborators:} (List here names of anyone you discussed
	problems or ideas for solutions with)
	
	% For homeworks 1 through n
	\textbf{No. of late days used on previous psets (not including Homework Zero): }\\
	\textbf{No. of late days used after including this pset: }
	
	
	\newpage
	

\textbf{Special notes for this homework:}
For your convenience, here's a list of some problems you can assume are in $\mathbf{P}$...
\begin{enumerate}
	\item Linear Programming: given a set of linear inequalities (like $2x_1 - 3x_2 \le 5$), is there a real solution to all of them?
	\item $2$-SAT: given a 2-CNF formula (an AND of clauses each an OR of at most 2 variables or their negations), is there an assignment of variables that makes it true?
	\item $2$-coloring: given a graph $G$, is there a way to color its vertices with 2 colors so that adjacent vertices have different colors?
	\item Shortest Path: given a graph $G$, vertices $s$ and $t$, and an integer $k$, is there a path of length at most $k$ from $s$ to $t$?
	\item Min Cut: given a graph $G$ and an integer $k$, is there a nonempty subset $S \subsetneq V$ with at most $k$ edges from $S$ to $\overline{S}$? 
\end{enumerate}
...and here's a list of some problems you can assume are $\mathbf{NP-complete}$:
\begin{enumerate}
	\item Integer Programming: given a set of linear inequalities (like $2x_1 - 3x_2 \le 5$), is there an integer solution to all of them?
	\item $3$-SAT: given a 3-CNF formula (an AND of clauses each an OR of at most 3 variables or their negations), is there an assignment of variables that makes it true?
	\item $3$-coloring: given a graph $G$, is there a way to color its vertices with 3 colors so that adjacent vertices have different colors?
	\item Longest Path: given a graph $G$, vertices $s$ and $t$, and an integer $k$, is there a path of length at least $k$ from $s$ to $t$? (A path is a sequence of \emph{distinct} vertices with each consecutive pair adjacent.)
	\item Max Cut: given a graph $G$ and an integer $k$, is there a nonempty subset $S \subsetneq V$ with at least $k$ edges from $S$ to $\overline{S}$? 
\end{enumerate}

	\subsection*{Questions}\label{questions}
Please solve the following problems. Some of these might be harder than
the others, so don't despair if they require more time to think or you
can't do them all. Just do your best. Also, you should only attempt the
bonus questions if you have the time to do so. If you don't have a proof
for a certain statement, be upfront about it. You can always explain
clearly what you are able to prove and the point at which you were
stuck. 
%Also, for a non bonus question, 
You can always simply write
\textbf{``I don't know''} and you will get 15 percent of the credit for
the problem. If you are stuck on this problem set, you can use Piazza to
send a private message to all staff.


\textbf{Note on reading the textbook:} If you are stuck on some of the
problems, try consulting the book to 1) understand the concepts the
question is referencing, and 2) review the way similar theorems are
proved in the book.


\newcommand{\HALT}{\mathrm{HALT}}

\textbf{Problem 0 (5 points):} True or False: I have completed the midterm 2 feedback survey. (True worth 5 points, False, or I don't know worth 0 points.)

\textbf{Problem 1.1 (10 points):} Suppose that you are in charge of
scheduling courses in computer science in University S. In University S,
computer science students wake up late, and have to work on their
startups in the afternoon, and take long weekends with their investors.
So you only have two possible slots: you can schedule a course either
Monday-Wednesday 11am-1pm or Tuesday-Thursday 11am-1pm.

Let \(SCHEDULE_S:\{0,1\}^* \rightarrow \{0,1\}\) be the function that
takes as input a list of courses \(L\) and a list of \emph{conflicts}
\(C\) (i.e., list of pairs of courses that cannot share the same time
slot) and outputs \(1\) if and only if there is a ``conflict free''
scheduling of the courses in \(L\) to the two slots, where no pair in
\(C\) is scheduled in the same time slot. (We model the course list
\(L\) as just a list of strings, with the conflict lists \(C\) as a list
of pairs of strings.)

Prove that \(SCHEDULE_S \in \mathbf{P}\). As usual, you do not have to
provide the full code to show that this is the case, and can describe
operations as a high level, as well as appeal to any data structures or
other results mentioned in the book or in lecture. Note that to show
that a function \(F\) is in \(\mathbf{P}\) you need to \textbf{(1)}
present an algorithm \(A\), 
\textbf{(2)} prove that \(A\) computes \(F\) in polynomial time, and
\textbf{(3)} prove that \(A\) runs in polynomial time. See footnote for hint.\footnote{Try to see how you model the setting
	mathematically, and whether it amounts to questions about objects we
	have looked at before.}

\textbf{Solution 1.1:}

\textbf{Problem 1.2 (10 points):} Consider the same question but now at
university H where there is a third course slot on Monday-Wednesday from
11pm till 1am. Let \(SCHEDULE_H:\{0,1\}^* \rightarrow \{0,1\}\) be the
function as above that takes as input a list of courses \(L\) and a list
of \emph{conflicts} \(C\) and outputs \(1\) if and only if there is a
``conflict free'' scheduling of the courses in \(L\) to the three slots
of \(H\). Prove that \(SCHEDULE_H\) is \(\mathbf{NP}\)-complete.

\textbf{Solution 1.2:}



\textbf{Problem 2} Recall that $\mathbf{P}$, $\mathbf{NP}$, and other complexity classes are sets of \emph{Boolean} functions $f: \{0,1\}^* \to \{0,1\}$; for instance, 3SAT returns 1 iff there exists a satisfying assignment to its input formula. In this problem, we want a polynomial-time algorithm to \emph{find} a satisfying assignment.

\textbf{Problem 2.1 (5 points)} If $\phi(x_0, x_1, \ldots, x_{n-1})$ is a 3-CNF formula and $i \in \{0,1\}$, let $\phi_i(x_1, \ldots, x_{n-1}) = \phi(i,x_1, \ldots, x_{n-1})$. Show that, given $3SAT(\phi_0)$ and $3SAT(\phi_1)$, you can calculate $3SAT(\phi)$.

\textbf{Problem 2.2 (15 points)} Suppose you have a polynomial-time algorithm $M$ that computes \textsc{3SAT}. Give a polynomial-time algorithm $M'$ that, on input a 3-CNF formula $\phi(x_0, x_1, \ldots, x_{n-1})$, returns a satisfying assignment if one exists, or 0 if none does.

\textbf{Problem 2.3 (Bonus, 0 points (hard))} Give an algorithm $M'$ with the following property: If it is true that $\textsc{3SAT} \in P$, then, on input a satisfiable 3-CNF formula $\phi(x_0, x_1, \ldots, x_{n-1})$, $M'$ returns a satisfying assignment in polynomial time. (If no satisfying assignment to its input formula exists, $M'$ may not halt in polynomial time.)

\textbf{Problem 2.4 (25 points)} Suppose you have a polynomial-time algorithm $M$ that computes Longest Path. Give a polynomial-time algorithm $M'$ that, on input a graph $G$, vertices $s$ and $t$, and an integer $k$, returns a path of length at least $k$ from $s$ to $t$ if one exists, or 0 if none does.


\textbf{Question 3.1 (15 points):} \textsc{$k$-1s-Positive-3SAT} is the following
function: the input is a CNF formula \(\varphi\) where each clause is
the OR of one to three variables (\emph{without negations}), and a
number \(k\in \mathbb{N}\). For example, the following is a valid input:
\(\left(\varphi = (x_5 \vee x_{2} \vee x_1) \wedge (x_1 \vee x_3) \wedge (x_2 \vee x_4 \vee x_0), k = 2\right)\).
The output is 1 if and only if there exists a
satisfying assignment to \(\varphi\) in which exactly \(k\) of the
variables get the value \(1\). For example, for the formula \(\varphi\)
above, \(\textsc{$k$-1s-Positive-3SAT}(\varphi,2)=1\) since the assignment \((1,1,0,0,0,0)\)
satisfies all the clauses. However \(\textsc{$k$-1s-Positive-3SAT}(\varphi,1)=0\) since there is
no single variable appearing in all clauses.

Prove that \textsc{$k$-1s-Positive-3SAT} is \(\mathbf{NP}\)-complete. See footnote for
hint.\footnote{\textbf{Hint:} You can use a direct reduction from
	\(3SAT\) or a reduction from a function that was shown to be
	\(\mathbf{NP}\)-complete in either the textbook, lecture, or section.
	If you go via the direct reduction route, you might want to try to
	transform a SAT instance on \(n\) variables \(x_0,\ldots,x_{n-1}\) to
	a \textsc{$k$-1s-Positive-3SAT} instance on \(2n\) variables \(w_0,\ldots,w_{2n}\) so that for
	every \(i\in [n]\), the variable \(w_{2i}\) will correspond to \(x_i\)
	and the variable \(w_{2i+1}\) will correspond to \(\neg x_i\). You can
	try to find ways to constrain a weight \(n\) solution so that exactly
	one of the variables \(w_{2i}\) and \(w_{2i+1}\) will equal \(1\) for
	every \(i \in [n]\).}

\textbf{Solution 3.1:}

\textbf{Question 3.2 (15 points):} In the \emph{employee recruiting
	problem} we are given a list of potential employees, each of which has
some subset of \(m\) potential skills, and a number \(k\). We want a team of \(k\) employees such that for every skill there is at least one member of the team with it.

For example, if Alice has the skills ``C programming'', ``NAND
programming'' and ``Solving Differential Equations'', Bob has the skills
``C programming'' and ``Solving Differential Equations'', and Charlie
has the skills ``NAND programming'' and ``Coffee Brewing'', then if we
want a team of $k=2$ people that covers all $m=4$ skills, we could hire
Alice and Charlie.

Define the function \(EMP\) s.t. on input the skills \(L\) of all
potential employees (in the form of a sequence \(L\) of \(n\) lists
\(L_1,\ldots,L_n\), each containing distinct numbers between \(0\) and
\(m\)), and a number \(k\), \(EMP(L,k)=1\) if and only if there is a
subset \(S\) of \(k\) potential employees such that for every skill
\(j\) in \([m]\), there is at least one employee in \(S\) that has the
skill \(j\).

Prove that \(EMP\) is \(\mathbf{NP}\)-complete. You can use the result
of the previous subquestion even if you didn't prove it. See also
footnote.\footnote{You can assume that lists (some of which could
	potentially be empty) are represented as strings in some reasonable
	way, and for every skill \(j\) there is some employee with the skill
	\(j\), and so in particular both \(n\) and \(m\) are smaller than the
	length of the input as a string of bits.}

\textbf{Solution 3.2:}

\textbf{Problem 4 (20 points):} Prove that the ``balanced variant'' \emph{BMC} of
the maximum cut problem is \(\mathbf{NP}\)-complete: given a graph $G$ and an integer $k$, $BMC(G,k) = 1$ iff there is a subset $S \subsetneq V$ of size $|V|/2$ with at least $k$ edges from $S$ to $\overline{S}$.

\textbf{Solution 4:}

\newcommand{\CSAT}{\mathrm{CubicSAT}}

\textbf{Problem 5.1 (20 points):} The input to the $\CSAT$ function is $m$ polynomials $P_0,\ldots,P_{m-1}$ on $n$ variables $x_0,\ldots,x_{n-1}$. Each polynomial is a cubic equation in the variables, i.e., the degree of every monomial is at most $3$. Also, the coefficients are integers in the range $\{-n,\ldots,n\}$. (Each such polynomial can be expressed as a list of $O(n^3)$ integers - why?). $\CSAT(P_0,\ldots,P_{m-1}) = 1$ iff there exist integers $a_0,\ldots,a_{n-1}$ such that for every $j \in [m]$, $P_j(a_0,\ldots,a_{n-1})=0$. Prove that $\CSAT$ is NP-hard.

\end{document}
