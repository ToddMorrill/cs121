\documentclass[11pt]{article}
\usepackage{fullpage}
\usepackage{amsfonts}
\usepackage{amssymb}
\usepackage{xcolor}

\newcommand{\F}{\mathbb{F}}
\newcommand{\np}{\mathop{\rm NP}}
\newcommand{\binom}[2]{{#1 \choose #2}}
\newcommand{\Z}{{\mathbb Z}}
\newcommand{\vol}{\mathop{\rm Vol}}
\newcommand{\conp}{\mathop{\rm co-NP}}
\newcommand{\atisp}{\mathop{\rm ATISP}}
\renewcommand{\vec}[1]{{\mathbf #1}}
\newcommand{\cupdot}{\mathbin{\mathaccent\cdot\cup}}
\newcommand{\mmod}[1]{\ (\mathrm{mod}\ #1)}

\setlength{\parskip}{\medskipamount}
\setlength{\parindent}{0in}
%\input{dansmacs}


\begin{document}

	\section*{CS 121 Homework 1: Fall
		2020}\label{cs-121-homework-zero-fall-2020}

	\textbf{Some policies:} (See the course policies page at \\ {\tt http://madhu.seas.harvard.edu/courses/Fall2020/policy.html} for the
	full policies.)

	\begin{itemize}
		\item
		{\bf Collaboration:} You may collaborate with other   students that are currently enrolled in
		this course
		%(or, in the case of homework zero, planning to enroll in this course)
		in brainstorming and thinking through approaches to
		solutions, but you should write the solutions on your own and may not
		share them with other students.
		\item
		{\bf Owning your solution:} Always make sure that you ``own'' your solutions to this and other problem
		sets. That is, you should always first grapple with the problems on
		your own, and even if you participate in brainstorming sessions, make
		sure that what you write down to submit reflects your own understanding of the solution.
		This is in your interest as it ensures you have a solid
		understanding of the course material, which will help in the midterms
		and final. Getting 80\% of the problem
		set questions right on your own will be much better to both your
		understanding and grade than getting 100\% of the questions by
		gathering hints from others without true understanding.
		\item
		{\bf Serious violations:} Sharing questions or solutions with anyone outside this course,
		including posting on outside websites, is a violation of the honor
		code policy. In particular, you may not get help from students or materials from past years of 			this or equivalent courses.
		\item
		{\bf Submission Format:} The submitted PDF should be typed and in the same format as
		ours. Please include the text of the problems and write
		\textbf{Solution X:} before your solution. Please mark in Gradescope
		the pages where
		the solution to each question appears. We may deduct points if you
		submit in a different format.
		\item {\bf Late Day Policy:} To give students some flexibility to manage your schedule, you are allowed a total of {\bf eight} late days through the semester, but you may not take more than {\bf two} late days on any single problem set.
	\end{itemize}

	\textbf{By writing my name here I affirm that I am aware of these policies
		and abided by them while working on this problem set:}

	\textbf{Your name:} %(Write name and HUID here)

	\textbf{Collaborators:} %(List here names of anyone you discussed problems or ideas for solutions with)

	% For homeworks 1 through n
	\textbf{No. of late days used on previous psets (not including Homework Zero): 0
}\\
	\textbf{No. of late days used after including this pset: %Fill in
}


	\newpage


	\subsection*{Questions}\label{questions}

{	\color{red}
Please solve the following problems. Some of these might be harder than
the others, so don't despair if they require more time to think or you
can't do them all. Just do your best. If you don't have a proof
for a certain statement, be upfront about it. You can always explain
clearly what you are able to prove and the point at which you were
stuck. Also you can always simply write
\textbf{``I don't know''} and you will get 15 percent of the credit for
this problem. If you are stuck on this problem set, you can use Piazza to
send a private message to all staff.

}

\newcommand{\Tower}{\mathrm{Tower}}
\textbf{Problem 1:} You may use a calculator/spreadsheet for both parts of this question.

\textbf{Problem 1.1 (10 points):} The 1977 Apple II personal computer had a processor
speed of 1.023 Mhz or about $10^6$ operations per second. In 2019 the world’s fastest supercomputer performed 93
“petaflops”  or $9.3 \times 10^{16}$
basic steps per second. For each of the following running times
(as functions of the input length $n$), compute for each of the two computers how
large an input it could handle in a week of computation, if it runs
an algorithm with that running time: (a) $n$ operations, (b) $n^2$ operations, (c) $10^6 n \log_2 n$ operations, (d) $2^n$ operations and (e) $\Tower(n)$ operations, where $\Tower(0) = 1$ and $\Tower(m) = 2^{\Tower(m-1)}$. Your answers can be approximate, but you should get the two most significant decimal digits right.

\textbf{Solution 1.1:} %(write your solution here)

After solving/showing work, please summarize your answers here:

\begin{center}
\begin{tabular}{ |c|c|c| }
 \hline
 Problem & Apple II length in a week & 2019 length in a week \\
 \hline
 a &  &  \\
 \hline
 b &  &  \\
 \hline
 c &  &  \\
 \hline
 d &  &  \\
 \hline
 e &  &  \\
 \hline
\end{tabular}
\end{center}

\textbf{Problem 1.2 (5 points):}
Typically the number of operations that the fastest computers can perform doubles every three years\footnote{For those comparing carefully with the previous problem: the Apple II was not the fastest computer of 1977.}.  So in 2022 we may expect computers performing 186 petaflops. How would you compare the largest input that computers can handle in 2022 vs. what they could handle in 2019 if they run an algorithm that makes the following number of operations:
(a) $n$ operations, (b) $n^2$ operations, (c) $10^6 n \log_2 n$ operations, (d) $2^n$ operations and (e) $\Tower(n)$ operations. For each case express your answer as ``The largest $n$ in 2022 (roughly) grows/shrinks by an additive/multiplicative factor of $X$ compared to 2019.'' for the best number $X$ and choice of ``additive'' and ``multiplicative'' you can determine.

\textbf{Solution 1.2:} %(write your solution here)

After solving/showing work, summarize answers here:

\begin{center}
\begin{tabular}{ |c|c|c|c| }
 \hline
 Problem & Grows or Shrinks & Multiplicative or Additive & Rough Factor \\
 \hline
 a &  &  & \\
 \hline
 b &  &  & \\
 \hline
 c &  &  & \\
 \hline
 d &  &  & \\
 \hline
 e &  &  & \\
 \hline
\end{tabular}
\end{center}

\textbf{Problem 2 (24 points):}
For each pair of functions  $F$ and $G$ below, determine
which\footnote{Note: the number that hold may be 0 or more than 1.} of the following relations hold: $F = O(G)$, $F = \Omega(G)$,
$F = o(G)$, and $F = \omega(G)$.
\begin{enumerate}
	\item $F(n) = n$, $G(n) = 100n$.
	\item $F(n) = n$ and $G(n) = \sqrt{n}$.
	\item $F(n) = n \log n$, $G(n) = 2^{(\log n)^2}$.
    \item $F(n) = 2^n$, $G(n) = 8^n$.
    \item $F(n) = \binom{n}{\lceil .1 n \rceil}$ and $G(n) = 2^{.5 n}$, where $\binom{n}k$ is the number of subsets of $[n]$ of size $k$, where $[n] = \{1, 2, \ldots, n\}$.

{\bf Hint:} You may use Stirling's approximation for this.
\end{enumerate}

\textbf{Solution 2:} %(write your answers here)


\textbf{Problem 3:} The Pigeonhole Principle, as understood by pigeons, asserts that if there more pigeons than pigeonholes and all pigeons are sitting in pigeonholes, then some pigeonhole contains at least two pigeons.

Mathematicians have their own pigeonhole principle (PHP) which states that if $A$ and $B$ are finite sets such that $|A| > |B|$ and $f:A\to B$ is a function then $f$ is not a one-to-one\footnote{Vocabulary note for those with different math backgrounds: ``one-to-one'' is a synonym of ``injective''; ``one-to-one correspondence'' is a synonym of ``bijective function'', and ``onto'' is a synonym of ``surjective''.} function.

\textbf{Problem 3.0 (0 points):} Why is the math PHP called that?

\textbf{Problem 3.1 (1 point):} Prove the mathematical PHP. (You may use any facts stated in Barak, Chapter 1.)



\textbf{Solution 3.1:} %(write your solution here)

\textbf{Problem 3.2 (15 points):}
Let $n \ge 10$ and $M < 2^{\sqrt{n}}$ be positive integers. Given $n$ integers $a_1,\ldots,  a_n$ with $a_i \in \{1, \ldots, M\}$, prove that there exist two non-empty disjoint sets S and T such that $\sum_{i \in S} a_i = \sum_{j \in T} a_j$. (Hint: Use the PHP. But if you do, tell us what sets $A$ and $B$ and function $f$ are you applying the PHP to.)

\textbf{Solution 3.2:} %(write your solution here)

\textbf{Problem 3.2 (Bonus, 0 points):} Same question as Problem 3.2, except now $a_i \in \{-M,\ldots,+M\}$.

\textbf{Problem 4:}

\newcommand{\N}{\mathbb{N}}

\textbf{Problem 4.1 (5 points):}
Show that there is a string
representation of directed graphs with vertex set $[n]$ having at most
 $10n$ edges that uses at most $1000n \log n$ bits.

More formally: Define, for every $n,m \in \N$, $G_{n,m}$ to be the set of directed graphs\footnote{In CS 121, unless specified otherwise, every graph is \emph{simple}: each edge has two \emph{distinct} vertices and no two edges are equal.} over the vertex
set $[n]$ with {\em at most} $m$ edges. Prove that for
every sufficiently large $n$, there exists a one-to-one function $E:G_{n, 10n} \to \{0,1\}^{1000n\log n}$.

\textbf{Note:} When you see a ``round'' constant such as \(10\),
\(100\), or \(1000\) in a problem set, it usually means that it was
chosen arbitrarily, and there is no particular significance to the
actual number. In particular, it may well be that you could come up with
such a scheme where \(E\) maps \(G_{n, 10n}\) to a string of
length at most \(cn\log n\) for some constant \(c\) that is significantly
smaller than \(1000\).

\textbf{Solution 4.1:} %(write your solution here)

\textbf{Problem 4.2 (10 points):}  Define $S_n$ to be the
set of one-to-one and onto functions mapping $[n]$ to $[n]$.
%Define $G_{2n,n}$ to be the set of directed graphs with vertex set $[2n]$ and exactly $n$ edges.
Prove that there is a one-to-one mapping from $S_n$ to $G_{2n,n}$.

\textbf{Solution 4.2:} %(write your solution here)

\textbf{Problem 4.3 (10 points):}
Show that the encoding length of Problem 4.1 can not be improved to $o(n \log n)$. Specifically, prove that
for every sufficiently large $n\in\N$, there is no one-to-one function
$E:G_{n, 10n} \to \{0,1\}^{.0001 n \log n}$.

\textbf{Solution 4.3:} %(write your solution here)



\end{document}
