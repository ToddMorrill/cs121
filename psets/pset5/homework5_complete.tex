\def\draft{1}

\documentclass[11pt]{article}   
\usepackage{fullpage}
\usepackage{amsfonts}
\usepackage{amssymb}
\usepackage{amsmath}
\usepackage{xcolor}
\usepackage{listings}

\newcommand{\F}{\mathbb{F}}
\newcommand{\np}{\mathop{\rm NP}}
%\newcommand{\binom}[2]{{#1 \choose #2}}
\newcommand{\Z}{{\mathbb Z}}
\newcommand{\vol}{\mathop{\rm Vol}}
\newcommand{\conp}{\mathop{\rm co-NP}}
\newcommand{\atisp}{\mathop{\rm ATISP}}
\renewcommand{\vec}[1]{{\mathbf #1}}
\newcommand{\cupdot}{\mathbin{\mathaccent\cdot\cup}}
\newcommand{\mmod}[1]{\ (\mathrm{mod}\ #1)}  

\setlength{\parskip}{\medskipamount}
\setlength{\parindent}{0in}
%\input{dansmacs}


\begin{document}
	
	\section*{CS 121 Homework 5: Fall
		2020}\label{cs-121-homework-zero-fall-2020}
	
	\textbf{Some policies:} (See the course policy at  {\tt http://madhu.seas.harvard.edu/courses/Fall2020/policy.html} for the
	full policies.)
	
	\begin{itemize}
		\item
		{\bf Collaboration:} You can collaborate with other students that are currently enrolled in
		this course (or, in the case of homework zero, planning to enroll in
		this course) in brainstorming and thinking through approaches to
		solutions but you should write the solutions on your own and cannot
		share them with other students. 
		\item
		{\bf Owning your solution:} Always make sure that you ``own'' your solutions to this other problem
		sets. That is, you should always first grapple with the problems on
		your own, and even if you participate in brainstorming sessions, make
		sure that you completely understand the ideas and details underlying
		the solution. This is in your interest as it ensures you have a solid
		understanding of the course material, and will help in the midterms
		and final. Getting 80\% of the problem
		set questions right on your own will be much better to both your
		understanding than getting 100\% of the questions through
		gathering hints from others without true understanding.
		\item
		{\bf Serious violations:} Sharing questions or solutions with anyone outside this course,
		including posting on outside websites, is a violation of the honor
		code policy. Collaborating with anyone except students currently
		taking this course or using material from past years from this or
		other courses is a violation of the honor code policy.
		\item
		{\bf Submission Format:} The submitted PDF should be typed and in the same format and
		pagination as ours. Please include the text of the problems and write
		\textbf{Solution X:} before your solution. Please mark in gradescope 
		the pages where
		the solution to each question appears. Points will be deducted if you
		submit in a different format.
		\item {\bf Late Day Policy:} To give students some flexibility to manage your schedule, you are allowed a net total of {\bf eight} late days through the semester, but you may not take more than {\bf two} late days on any single problem set. No exceptions to this policy. 
	\end{itemize}
	
	\textbf{By writing my name here I affirm that I am aware of all policies
		and abided by them while working on this problem set:}
	
	\textbf{Your name:} Todd Morrill
	
	\textbf{Collaborators:} (List here names of anyone you discussed
	problems or ideas for solutions with)
	
	% For homeworks 1 through n
	\textbf{No. of late days used on previous psets (not including Homework Zero): } 1\\
	\textbf{No. of late days used after including this pset: } 1
	
	
	\newpage
	
	\subsection*{Questions}\label{questions}


Please solve the following problems. Some of these might be harder than
the others, so don't despair if they require more time to think or you
can't do them all. Just do your best. Also, you should only attempt the
bonus questions if you have the time to do so. If you don't have a proof
for a certain statement, be upfront about it. You can always explain
clearly what you are able to prove and the point at which you were
stuck. 
%Also, for a non bonus question, 
You can always simply write
\textbf{``I don't know''} and you will get 15 percent of the credit for
the problem. If you are stuck on this problem set, you can use Piazza to
send a private message to all staff.


\textbf{Note on reading the textbook:} If you are stuck on some of the
problems, try consulting the book to 1) understand the concepts the
question is referencing, and 2) review the way similar theorems are
proved in the book.

\newcommand{\HALT}{\mathrm{HALT}}
\newcommand{\SPACEHOG}{\textrm{SPACE-HOG}}

\newpage

\textbf{Problem 1:}

\textbf{Problem 1.1 (4 points):} Prove that the function $f_1:\{0,1\}^* \to \{0,1\}$ for which $f_1(x) = 1$ iff $x = 1001111$ is computable.

\textbf{Solution 1.1}

$f_1$ can be computed by simply checking if $x = 1001111$, which can be done in constant time using a high-level programming language or equivalently using a TM. If $x = 1001111$ the procedure returns 1 (i.e. $f_1(x) = 1$). The procedure returns 0 otherwise (i.e. $f_1(x) = 0$).

\newpage

\textbf{Problem 1.2 (6 points):} Prove that the function $f_2:\{0,1\}^* \to \{0,1\}$ for which $f_2(x) = 1$ iff $x$ is an encoding of a Turing Machine that calculates $f_1$ is uncomputable.

\textbf{Solution 1.2}

The proof is by Rice's theorem.

\textbf{Nontriviality:} There exist two different TM's $M$, $M'$ such that $f(M) = 1$ and $f(M') = 0$. For example, $M$ could be the TM that computes $f_1$ as described in \textbf{solution 1.1}. $M'$ could be the TM that computes the constant one function. 

\textbf{Semantic Property:} Recall that $f_2(M) = 1$ iff $M$ is an encoding of a Turing Machine that calculates $f_1$. If $M$ and $M'$ are functionally equivalent TMs then $M(x) = M'(x)$ for all $x \in \{0, 1\}^*$. This would imply that $f_2(M) = 1$ iff $f_2(M') = 1$.

From this we conclude that $f_2$ is uncomputable.

\newpage

\textbf{Problem 1.3 (5 points):} Prove or disprove: There exists a constant $C$ and an uncomputable function $f:\{0,1\}^* \to \{0,1\}$ such that the number of values of $x$ for which $f(x)=1$ is at most $C$. %(I.e., $|\{x \in \{0,1\}^* | f(x) = 1\}\}| \leq C$.

\textbf{Solution 1.3}

Suppose for the sake of contradiction that there was a constant $C$ and an uncomputable function $f:\{0,1\}^* \to \{0,1\}$ such that the number of values of $x$ for which $f(x)=1$ is at most $C$. %(I.e., $|\{x \in \{0,1\}^* | f(x) = 1\}\}| \leq C$.

If this were the case, we could create a finite length array of these values for which $f(x) = 1$ and check if $x$ is equivalent to any of the bit strings in the array (as we did for one particular value of $x$ in \textbf{solution 1.1}), which could be done in time $O(C)$. If $x$ matches any of the values in the array, the procedure returns 1. The procedure returns 0 otherwise. But this would imply that $f$ is computable, which contradicts the fact that $f$ was given to be uncomputable.

From this we conclude that there does not exist a constant $C$ and an uncomputable  function $f:\{0,1\}^* \to \{0,1\}$ such that the number of values of $x$ for which $f(x)=1$ is at most $C$.

\newpage

\textbf{Problem 2:} For each of the following problems, write either ``can be proven uncomputable by Rice's theorem'' (i.e. the function is semantic and nontrivial) or ``can't be proven uncomputable by Rice's theorem'' for 2 points (no justification necessary for this part). Then, write either ``computable'' or ``uncomputable'', with brief justification, for a variable number of points. (If you answered ``can be proven uncomputable by Rice's theorem'', justification is why the function's semantic and nontrivial.)

\textbf{Problem 2.1 (2+3 points):}  $f(M)=1$ iff there is an input $x$ such that $M(x) = 1$.

\textbf{Solution 2.1}

$f$ can be proven uncomputable by Rice's theorem. $f$ is uncomputable.

\textbf{Nontriviality:} There exist two different TM's $M$, $M'$ such that $f(M) = 1$ and $f(M') = 0$. For example, $M$ could be the TM that computes the constant 1 function. $M'$ could be the TM that computes the constant zero function. 

\textbf{Semantic Property:} Recall that $f(M) = 1$ iff there is an input $x$ such that $M(x) = 1$. If $M$ and $M'$ are functionally equivalent TMs then $M(x) = M'(x)$ for all $x \in \{0, 1\}^*$. This would imply that $f(M) = 1$ iff $f(M') = 1$.

From this we conclude that $f$ is uncomputable.

\newpage

\textbf{Problem 2.2 (2+3 points):} $f(M) = 1$ iff any of $M$'s instructions include ``move left''.

\textbf{Solution 2.2}

$f$ can't be proven uncomputable by Rice's theorem. $f$ is computable.

$M$ encodes its transition table, which is finite by definition. This transition table can be searched for the action ``move left'' in time $O(n)$ where $n$ is the number of entries in the transition table.

Correctness: If the search procedure finds the action ``move left'' it returns 1 (i.e. $f(M) = 1$). If the procedure does not find the action ``move left'', it returns 0 (i.e. $f(M) = 0$).

From this we conclude that $f$ is computable.

\newpage

\textbf{Problem 2.3 (2+Bonus 0 points):} $f(M) = 1$ iff there exists an input for which $M$ ever moves left.

\textbf{Solution 2.3}

$f$ can't be proven uncomputable by Rice's theorem because $f$ isn't semantic. $f$ is computable.

$f$ can be equated to a DFA that reads input from left to right and if it encounters an action to ``move left'' it enters into an accept state (i.e. $f(M) = 1$). If $M$ halts before moving left, then the DFA enters into a reject state (i.e. $f(M) = 0$).

\newpage

\textbf{Problem 2.4 (2+4 points):}  $f(M,n) = 1$ iff it is the case that for every input of length $n$, $M$ runs in time at most $2^n$.

\textbf{Solution 2.4}

$f$ can't be proven uncomputable by Rice's theorem because $f$ isn't semantic. $f$ is computable.

Since $n$ is a fixed number, there are a finite number of bit strings $x \in \{0, 1\}^n$. We can run the Universal TM on each of these bit strings (i.e. $U(M, x)$) and check if each one halts in at most $2^n$ steps (adding extra steps for the overhead of running the simulation). If $M(x)$ halts in at most $2^n$ steps for all $x \in \{0, 1\}^n$ then the procedure returns 1 (i.e. $f(M,n) = 1$). Otherwise, the procedure returns 0 (i.e. $f(M,n) = 1$).

\newpage

\textbf{Problem 2.5 (2+Bonus (hard) 0 points):}  $f(M) = 1$ iff it is the case that for every integer $n$ and every input of length $n$, $M$ runs in time at most $2^n$.

\textbf{Solution 2.5}

$f$ can't be proven uncomputable by Rice's theorem because $f$ isn't semantic.

\newpage

\textbf{Problem 2.6 (2+3 points):}  $f(M) = 1$ iff it is the case that for every $x$, $M(x) = 1$ if and only if $x = x^R$. ($x^R$ denotes the reverse of $x$.)

\textbf{Solution 2.6}

$f$ can be proven uncomputable by Rice's theorem. $f$ is uncomputable.

\textbf{Nontriviality:} There exist two different TM's $M$, $M'$ such that $f(M) = 1$ and $f(M') = 0$. For example, $M$ could be the TM that we saw in class that computes the PALINDROME function. $M'$ could be the TM that computes the constant one function. 

\textbf{Semantic Property:} Recall that $f(M) = 1$ iff it is the case that for every $x$, $M(x) = 1$ if and only if $x = x^R$. If $M$ and $M'$ are functionally equivalent TMs then $M(x) = M'(x)$ for all $x \in \{0, 1\}^*$. This would imply that $f(M) = 1$ iff $f(M') = 1$.

From this we conclude that $f$ is uncomputable.

\pagebreak


\textbf{Problem 3:}

\textbf{Problem 3.1 (5 points):} The ``configuration'' of a Turing Machine at a time $t$ when it is in the middle of a computation is information that determines its future operation. Specifically, for a machine $M$ with $k$ states and alphabet $\Sigma$, its configuration is the triple $(q,i,x)$ where $q \in [k]$ is its current state,  $i \in \mathbb{N}$ is the position of the tape head, and $x \in \Sigma^*$ is the string of cells on the tape between $\triangleright$ and the first $\phi$. (Assume the TM never writes a $\phi$ or overwrites any but the leftmost $\phi$.) Prove that if, on some input $w$, a machine $M$ reaches the same configuration twice, i.e. $C_1 = (q_1,i_1,x_1)$ if the configuration at time $t_1$ and $C_2 = (q_2,i_2,x_2)$ is the configuration at time $t_2$ and $C_1 = C_2$, then $M$ never halts on input $w$.

\textbf{Solution 3.1}

The proof is by contradiction. Suppose that a TM $M$ that reaches the same configuration $C_1 = C_2$ at time $t_1$ and $t_2$ on input $w$ does halt. When the TM transitions from configuration $C_1$ to $C_2$, it is really back in configuration $C_1$ by the equivalence in $C_1$ and $C_2$. Since TM's have a finite transition table and behave deterministically, the machine will again transition from $C_1$ to $C_2$, as it did the first time, in some finite number of steps. But this means that the machine will continue looping back to $C_1$ and in particular, it means that the TM will not halt on input $w$. But this is a contradiction, since we assumed that $M$ would halt on input $w$.

From this we conclude that if a machine $M$ reaches the same configuration twice (i.e. $C_1 = C_2$) at two different time steps then $M$ never halts on input $w$.

\newpage

\textbf{Problem 3.2 (15 points):} Let $\SPACEHOG(M,x,s) = 1$ if $M$ is a Turing Machine that on input $x$ uses more than $s$ bits of space (i.e., tape head ever reaches the $s+1$th cell of the tape). 
Prove that $\SPACEHOG$ is computable.

\textbf{Solution 3.2}

The proof is by the Pigeon-Hole Principle (PHP). The number of configurations of $M$ is $C = k \times s \times |\Sigma|^s$, where $k$ is the number of states that $M$ has, $s$ is the index $i$ such that $M$ uses at most $s$ slots on the tape, and $|\Sigma|^s$ is the number of configurations of the tape on alphabet $\Sigma$ with length $s$.

We define a procedure below to compute $\SPACEHOG$.
\begin{lstlisting}
def compute_spacehog(M, x, s):
    C = k x s x |Sigma|^s
    for _ in range(C):
        Simulate one step of M(x) using the universal TM
        if the index i>s:
            return 1
    return 0
\end{lstlisting}

Correctness: When $\SPACEHOG(M,x,s) = 1$, compute\_spacehog(M, x, s) = 1, because it checks if $M$'s tape has used more then $s$ slots of the tape at every step and returns 1 if this happens. When $\SPACEHOG(M,x,s) = 0$, compute\_spacehog(M, x, s) = 0 for one of two possible reasons: 1) $M(x)$ halts before $i>s$ or 2) $M$ has potentially cycled through all of its unique configurations and if it hasn't halted by that point, by the PHP, $M$ will visit a  configuration two times. By \textbf{solution 3.1} $M$ will not halt because it will visit a configuration twice.

From this we conclude that that $\SPACEHOG$ is computable.

\newpage

\textbf{Problem 4 (20 points):} Recall that $\HALT(M,x) = 1$ if $M$ is the encoding of a Turing Machine and $M$ halts on $x$, and $\HALT(M,x) = 0$ otherwise. 

For integer $C$, let $\HALT_C(M,x) = 1$ if and only if both $\HALT(M,x) = 1$ and $M$ is a Turing Machine with at most $C$ states. 
 
Prove that there exists a constant $C$ such that $\HALT_C$ is uncomputable.

\textbf{Solution 4}

The proof is by reduction from $\HALT$ to $\HALT_C$.

Recall that the Universal TM $U$ can evaluate another TM $M$ on $X$ using a constant number of states, which we denote $C_U$. In particular, $U(M, x) = M(x)$. Suppose for the sake of contradiction that for all constants $C$, $\HALT_C$ is computable, which means that $\HALT_{C_U}$ is computable. With this in hand, we can compute $\HALT$ as follows.

\begin{lstlisting}
def compute_halt(M, x):
    return compute_halt_c_u(U, (M, x))
\end{lstlisting}

Correctness: When $\HALT(M, x) = 1$, compute\_halt(M, x) = 1 because compute\_halt returns 1 iff compute\_halt\_c\_u(U, (M, x)) = 1, which only returns 1 if $M(x)$ halts. compute\_halt\_c\_u satisfies the condition that $U$ is a TM with at most $C_U$ states when fed the Universal TM. When $\HALT(M, x) = 0$, compute\_halt(M, x) = 0 because compute\_halt returns 0 iff compute\_halt\_c\_u(U, (M, x)) = 0, which returns 0 if $M(x)$ does not halt.

This implies that we can compute $\HALT$, which we know to be uncomputable. From this we conclude that compute\_halt\_c\_u is uncomputable. But this is a contradiction since we assumed that for all constants $C$, $\HALT_C$ is computable. This shows that there exists a constant $C$ such that $\HALT_C$ is uncomputable.
\newpage


\textbf{Problem 5 (20 points):} Let $f(M) = 1$ if $M$ is a Turing Machine that ever writes the symbol $@$ on its tape on input the empty string. Let $f(M) = 0$ in all other cases. Show that $f(M)$ is uncomputable. (Hint: reduce from the halting problem.)

\textbf{Solution 5}

The proof is by reduction from $\HALT$. Suppose for the sake of contradiction that we could compute $f$ with a procedure called compute\_f. With this procedure in hand, we could compute $\HALT$ as follows.

\begin{lstlisting}
def compute_halt(M, x):
    M' = M with all occurrences of @ replaced with !
    def N(y):
        M'(x)
        Write @ on the tape
    return compute_f(N)
\end{lstlisting}

The line $M' = M$ with all occurrences of @ replaced with ! is required to ensure that compute\_f doesn't return 1 in the case where $M$ writes an @ on the tape but doesn't halt. This replacement can be done in $O(n)$ time in the size of the transition table.

Correctness: When $HALT(M, x) = 1$, compute\_halt(M, x) = 1. If M'(x), which is equivalent to $M(x)$ up to a change of one character ($@ \to !$), halts then compute\_f(N) = 1 because $N$ writes an @ on the tape when input the empty string. When $HALT(M, x) = 0$, compute\_halt(M, x) = 0. If M'(x) does not halt then compute\_f(N) = 0 because $N$ never writes an @ on the tape.

This implies that there exists a procedure to compute $\HALT$, which we know to be uncomputable. Our only assumption was that there existed a procedure called compute\_f, which must be false and we conclude that $f(M)$ is uncomputable.


\newpage


\textbf{Problem 6:}

\textbf{Problem 6.1 (Bonus, 0 points):} Give an example of functions $f,g,F,G$ such that $f(n) = O(F(n))$, $g(n) = O(G(n))$ but $f(g(n)) \not\in O(F(G(n)))$. 
\begin{align*} 
f(n) &= n & g(n) &= \frac{1}{n^2}\\
F(n) &= n^3 & G(n) &= \frac{1}{n}
\end{align*}
\begin{align*} 
f((g(n)) &= \frac{1}{n^2} \notin O(F(G(n)) = O(\frac{1}{n^3})
\end{align*}


\newpage

\textbf{Problem 6.2 (10 points):}  Prove that if $f(n) = O(n^c)$ and $g(n) = O(n^d)$ for $c,d > 0$ then $f(g(n)) = O(n^{cd})$. 

\textbf{Solution 6.2}

If $f(n) = O(n^c)$ and $g(n) = O(n^d)$ then there exist constants $c_1$ and $c_2$ such that $f(n) \leq c_1 n^c$ and $g(n) \leq c_2 n^d$ for all $n > n_0$ by the definition of Big-O.

Composing f and g, we have $f(g(n)) \leq c_1(c_2 n^d)^c = c_1 \cdot c_2^c \cdot n^{cd} = c_3 \cdot n^{cd}$ for all $n > n_0$, where $c_3 = c_1 \cdot c_2^c$. But this is precisely the definition of Big-O and therefore $f(g(n)) = O(n^{cd})$. From this we conclude that if $f(n) = O(n^c)$ and $g(n) = O(n^d)$ for $c,d > 0$ then $f(g(n)) = O(n^{cd})$.

\newpage

The contrast between 6.1 and 6.2 may mean that the following problem, which you solved in sections, was harder than you thought

\textbf{Problem 6.3 (Bonus, 0 points):}  Prove that if $A:\{0,1\}^* \to \{0,1\}^*$ and $B:\{0,1\}^* \to \{0,1\}^*$ are polynomial time computable functions then their composition $A(B(x))$ is also polynomial time computable.

\textbf{Solution 6.3}

The proof follows from \textbf{solution 2.2} from homework 4, which showed that the output from one TM could be fed to another with very little computational overhead, in particular in time $O(T_B)$ where $T_B$ is the time complexity of $B$. The proof also follows from the results demonstrated in \textbf{solution 6.2}, which showed that the composition of two polynomial time functions is still polynomial.

\end{document}
